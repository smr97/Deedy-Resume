%%%%%%%%%%%%%%%%%%%%%%%%%%%%%%%%%%%%%%%
% Deedy - One Page Two Column Resume
% LaTeX Template
% Version 1.2 (16/9/2014)
%
% Original author:
% Debarghya Das (http://debarghyadas.com)
%
% Original repository:
% https://github.com/deedydas/Deedy-Resume
%
% IMPORTANT: THIS TEMPLATE NEEDS TO BE COMPILED WITH XeLaTeX
%
% This template uses several fonts not included with Windows/Linux by
% default. If you get compilation errors saying a font is missing, find the line
% on which the font is used and either change it to a font included with your
% operating system or comment the line out to use the default font.
% 
%%%%%%%%%%%%%%%%%%%%%%%%%%%%%%%%%%%%%%
% 
% TODO:
% 1. Integrate biber/bibtex for article citation under publications.
% 2. Figure out a smoother way for the document to flow onto the next page.
% 3. Add styling information for a "Projects/Hacks" section.
% 4. Add location/address information
% 5. Merge OpenFont and MacFonts as a single sty with options.
% 
%%%%%%%%%%%%%%%%%%%%%%%%%%%%%%%%%%%%%%
%
% CHANGELOG:
% v1.1:
% 1. Fixed several compilation bugs with \renewcommand
% 2. Got Open-source fonts (Windows/Linux support)
% 3. Added Last Updated
% 4. Move Title styling into .sty
% 5. Commented .sty file.
%
%%%%%%%%%%%%%%%%%%%%%%%%%%%%%%%%%%%%%%%
%
% Known Issues:
% 1. Overflows onto second page if any column's contents are more than the
% vertical limit
% 2. Hacky space on the first bullet point on the second column.
%
%%%%%%%%%%%%%%%%%%%%%%%%%%%%%%%%%%%%%%


\documentclass[]{smresume}
\usepackage{fancyhdr}
 
\pagestyle{fancy}
\fancyhf{}
 
\begin{document}

%%%%%%%%%%%%%%%%%%%%%%%%%%%%%%%%%%%%%%
%
%     LAST UPDATED DATE
%
%%%%%%%%%%%%%%%%%%%%%%%%%%%%%%%%%%%%%%
\lastupdated

%%%%%%%%%%%%%%%%%%%%%%%%%%%%%%%%%%%%%%
%
%     TITLE NAME
%
%%%%%%%%%%%%%%%%%%%%%%%%%%%%%%%%%%%%%%
\namesection{Saurabh}{Raje}{\href{mailto:saurabh.mraje@gmail.com}{saurabh.mraje@gmail.com}
}

%%%%%%%%%%%%%%%%%%%%%%%%%%%%%%%%%%%%%%
%
%     COLUMN ONE
%
%%%%%%%%%%%%%%%%%%%%%%%%%%%%%%%%%%%%%%

\begin{minipage}[t]{0.33\textwidth} 

%%%%%%%%%%%%%%%%%%%%%%%%%%%%%%%%%%%%%%
%     EDUCATION
%%%%%%%%%%%%%%%%%%%%%%%%%%%%%%%%%%%%%%

\section{Education} 

\subsection{BITS Pilani}
\descript{BE in Computer Science}
\location{May 2019 | Pilani, India}
\subsectionsep
%\subsection{Cornell University}
%\descript{BS in Computer Science}
%\location{May 2014 | Ithaca, NY}
%College of Engineering \\
%Magna Cum Laude\\
%\location{ Cum. GPA: 3.83 / 4.0 \\
%Major GPA: 3.9 / 4.0}

\subsection{Delhi Public School}
\location{July 2015|  Gurgaon, India}
\sectionsep

%%%%%%%%%%%%%%%%%%%%%%%%%%%%%%%%%%%%%%
%     LINKS
%%%%%%%%%%%%%%%%%%%%%%%%%%%%%%%%%%%%%%

\section{Links} 
Blog:// \href{https://smr97.github.io/categories/}{\bf smr97.github.io} \\
Github:// \href{https://github.com/smr97}{\bf smr97} \\
LinkedIn://  \href{https://www.linkedin.com/in/saurabhmraje/}{\bf saurabhmraje} \\
\sectionsep

%%%%%%%%%%%%%%%%%%%%%%%%%%%%%%%%%%%%%%
%     COURSEWORK
%%%%%%%%%%%%%%%%%%%%%%%%%%%%%%%%%%%%%%

\section{Coursework}
%\subsection{Graduate}
%Advanced Machine Learning \\
%Open Source Software Engineering \\
%Advanced Interactive Graphics \\
%Compilers + Practicum \\
%Cloud Computing \\
%Evolutionary Computation \\
%Defending Computer Networks \\
%Machine Learning \\
%\sectionsep

\subsection{Undergraduate}
Parallel Computing \\
Neural Networks and Fuzzy Logic \\
Operating Systems \\
Compiler construction \\
Data Structures and Algorithms \\
Database Systems \\
Object Oriented Programming \\
Principles of Programming Languages \\
Computer Networks \\
\sectionsep

%%%%%%%%%%%%%%%%%%%%%%%%%%%%%%%%%%%%%%
%     SKILLS
%%%%%%%%%%%%%%%%%%%%%%%%%%%%%%%%%%%%%%

\section{Skills}
\subsection{Programming Languages}
Rust \textbullet{} Java \textbullet{} Python \textbullet{} 
C \textbullet{} C++ 
\subsectionsep

\subsection{ML frameworks}
Tensorflow \textbullet{} Pytorch \textbullet{} Caffe
\subsectionsep

\subsection{HPC frameworks}
OpenMPI \textbullet{} OpenMP \textbullet{} CUDA
\subsectionsep

\subsection{Human languages}
Marathi \textbullet{} Hindi \textbullet{} English \textbullet{} French
\sectionsep

%%%%%%%%%%%%%%%%%%%%%%%%%%%%%%%%%%%%%%
%     AWARDS
%%%%%%%%%%%%%%%%%%%%%%%%%%%%%%%%%%%%%%

%\section{Awards} 
%\begin{tighttabular}%{rll}
%2018 & 1\textsuperscript{st} & Best Poster, IBM Research\\
%    2018 & 1\textsuperscript{st} & Hack.Banglore, Mercedes-Benz\\
%\end{tighttabular}
%\sectionsep

%%%%%%%%%%%%%%%%%%%%%%%%%%%%%%%%%%%%%%
%
%     COLUMN TWO
%
%%%%%%%%%%%%%%%%%%%%%%%%%%%%%%%%%%%%%%

\end{minipage} 
\hfill
\begin{minipage}[t]{0.66\textwidth} 

%%%%%%%%%%%%%%%%%%%%%%%%%%%%%%%%%%%%%%
%     EXPERIENCE
%%%%%%%%%%%%%%%%%%%%%%%%%%%%%%%%%%%%%%

\section{Experience}
\runsubsection{IBM Research}
\descript{| Research Engineer }
\location{August 2019 – Present | Delhi, India}
\vspace{\topsep} % Increases spacing if they overlap
\begin{tightemize}
\item Working with the HPC group to optimise deep language models (BERT).
\item Co-developed a novel technique (PoWER-BERT) to accelerate BERT model.
\item Also working on integrating this into IBM's \emph{\href{https://www.ibm.com/in-en/cloud/watson-natural-language-understanding}{Watson NLU}} product.
\end{tightemize}
\subsectionsep

\runsubsection{ETH Zurich - \emph{\href{https://spcl.inf.ethz.ch/}{SPCL lab}}}
\descript{| Scientific Assistant }
\location{March 2019 - August 2019 | Zurich, Switzerland}
%\vspace{\topsep} % Hacky fix for awkward extra vertical space
\begin{tightemize}
\item Used the \emph{\href{http://spcl.inf.ethz.ch/Research/DAPP/}{DACE}} domain specific language to accelerate \textbf{deep learning}.
\item Built a Tensorflow graph parser that would generate code for popular math operations on heterogenous computing platforms.
\item Added graph transformations to language IR to generate faster kernels.
\end{tightemize}
\subsectionsep

\runsubsection{IBM Research}
\descript{| Research Intern }
\location{May 2018 – August 2018 | Delhi, India}
\begin{tightemize}
\item Worked on training \textbf{deep neural networks} under memory constraints.
\item Implemented variable batch sizing to reduce training time by 20\%. 
\item Developed a scalable optimization algorithm for \textbf{Tensor Tucker Decomposition}.
\end{tightemize}
\subsectionsep

\runsubsection{INRIA}
\descript{| Research Intern }
\location{September 2018 – February 2019 | Grenoble, France}
\begin{tightemize}
\item Worked on a new parallelization library \emph{\href{http://datamove.imag.fr/frederic.wagner/rayon-adaptive.html}{Rayon Adaptive}} for the \textbf{Rust} language.
\item This automatically parallelizes functional code using \textbf{adaptive task splitting}. 
\item Contributed to a visualisation framework \emph{\href{http://datamove.imag.fr/frederic.wagner/rayon-logs.html}{Rayon-Logs}} that shows parallel runs.
\end{tightemize}
\subsectionsep

% \runsubsection{UST Global}
% \descript{| Software Engineering Intern }
% \location{May 2017 – July 2017 | Trivandrum, India}
% \begin{tightemize}
% \item Developed a decentralised malware detection engine.item Trained a \textbf{deep belief network} to classify EXE files
% \item Used the \textbf{Ethereum} network for decentralised consensus among the learners.
% \item This system achieved about 89\% accuracy in detecting malware.
% \end{tightemize}

%%%%%%%%%%%%%%%%%%%%%%%%%%%%%%%%%%%%%%
%    Projects
%    Data structure design
% 
%%%%%%%%%%%%%%%%%%%%%%%%%%%%%%%%%%%%%%

\section{Projects}
\runsubsection{Pedestrian detection system}
\descript{| Mercedes Benz Research}
%\location{Jan 2014 – Jan 2015 | Ithaca, NY}
\begin{tightemize}
\item \textbf{Won} the Daimler autonomous driving hackathon with a CNN model. 
\item Presented our solution at the \textbf{Mobile World Congress}.
%\item Was invited by Daimler to present this at the \textbf{Mobile World Congress}.
\end{tightemize}

\runsubsection{On-board computer for a nanosatellite}
\descript{| \href{http://www.team-anant.org/}{Team Anant}}
\begin{tightemize}
    \item Lead a student research initiative to build a computer for a nanosatellite.
    \item Wrote a parallel scheduler to run various mission critical modules.
    \item Wrote some kernel device drivers for CMOS sensors on an I2C bus.
    \item The satellite is scheduled for \textbf{launch by ISRO} in 2020.
\end{tightemize}
\section{Publications} 
\end{minipage}
%%%%%%%%%%%%%%%%%%%%%%%%%%%%%%%%%%%%%%
%     PUBLICATIONS
%%%%%%%%%%%%%%%%%%%%%%%%%%%%%%%%%%%%%%
\vspace{13pt}
\bibliographystyle{abbrv}
\bibliography{publications}
\nocite{*}
\end{document}  \documentclass[]{article}
